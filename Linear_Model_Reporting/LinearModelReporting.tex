\documentclass[]{article}
\usepackage{lmodern}
\usepackage{amssymb,amsmath}
\usepackage{ifxetex,ifluatex}
\usepackage{fixltx2e} % provides \textsubscript
\ifnum 0\ifxetex 1\fi\ifluatex 1\fi=0 % if pdftex
  \usepackage[T1]{fontenc}
  \usepackage[utf8]{inputenc}
\else % if luatex or xelatex
  \ifxetex
    \usepackage{mathspec}
  \else
    \usepackage{fontspec}
  \fi
  \defaultfontfeatures{Ligatures=TeX,Scale=MatchLowercase}
\fi
% use upquote if available, for straight quotes in verbatim environments
\IfFileExists{upquote.sty}{\usepackage{upquote}}{}
% use microtype if available
\IfFileExists{microtype.sty}{%
\usepackage{microtype}
\UseMicrotypeSet[protrusion]{basicmath} % disable protrusion for tt fonts
}{}
\usepackage[margin=1in]{geometry}
\usepackage{hyperref}
\hypersetup{unicode=true,
            pdftitle={LinearModelReporting.R},
            pdfauthor={GlennTattersall},
            pdfborder={0 0 0},
            breaklinks=true}
\urlstyle{same}  % don't use monospace font for urls
\usepackage{color}
\usepackage{fancyvrb}
\newcommand{\VerbBar}{|}
\newcommand{\VERB}{\Verb[commandchars=\\\{\}]}
\DefineVerbatimEnvironment{Highlighting}{Verbatim}{commandchars=\\\{\}}
% Add ',fontsize=\small' for more characters per line
\usepackage{framed}
\definecolor{shadecolor}{RGB}{248,248,248}
\newenvironment{Shaded}{\begin{snugshade}}{\end{snugshade}}
\newcommand{\KeywordTok}[1]{\textcolor[rgb]{0.13,0.29,0.53}{\textbf{#1}}}
\newcommand{\DataTypeTok}[1]{\textcolor[rgb]{0.13,0.29,0.53}{#1}}
\newcommand{\DecValTok}[1]{\textcolor[rgb]{0.00,0.00,0.81}{#1}}
\newcommand{\BaseNTok}[1]{\textcolor[rgb]{0.00,0.00,0.81}{#1}}
\newcommand{\FloatTok}[1]{\textcolor[rgb]{0.00,0.00,0.81}{#1}}
\newcommand{\ConstantTok}[1]{\textcolor[rgb]{0.00,0.00,0.00}{#1}}
\newcommand{\CharTok}[1]{\textcolor[rgb]{0.31,0.60,0.02}{#1}}
\newcommand{\SpecialCharTok}[1]{\textcolor[rgb]{0.00,0.00,0.00}{#1}}
\newcommand{\StringTok}[1]{\textcolor[rgb]{0.31,0.60,0.02}{#1}}
\newcommand{\VerbatimStringTok}[1]{\textcolor[rgb]{0.31,0.60,0.02}{#1}}
\newcommand{\SpecialStringTok}[1]{\textcolor[rgb]{0.31,0.60,0.02}{#1}}
\newcommand{\ImportTok}[1]{#1}
\newcommand{\CommentTok}[1]{\textcolor[rgb]{0.56,0.35,0.01}{\textit{#1}}}
\newcommand{\DocumentationTok}[1]{\textcolor[rgb]{0.56,0.35,0.01}{\textbf{\textit{#1}}}}
\newcommand{\AnnotationTok}[1]{\textcolor[rgb]{0.56,0.35,0.01}{\textbf{\textit{#1}}}}
\newcommand{\CommentVarTok}[1]{\textcolor[rgb]{0.56,0.35,0.01}{\textbf{\textit{#1}}}}
\newcommand{\OtherTok}[1]{\textcolor[rgb]{0.56,0.35,0.01}{#1}}
\newcommand{\FunctionTok}[1]{\textcolor[rgb]{0.00,0.00,0.00}{#1}}
\newcommand{\VariableTok}[1]{\textcolor[rgb]{0.00,0.00,0.00}{#1}}
\newcommand{\ControlFlowTok}[1]{\textcolor[rgb]{0.13,0.29,0.53}{\textbf{#1}}}
\newcommand{\OperatorTok}[1]{\textcolor[rgb]{0.81,0.36,0.00}{\textbf{#1}}}
\newcommand{\BuiltInTok}[1]{#1}
\newcommand{\ExtensionTok}[1]{#1}
\newcommand{\PreprocessorTok}[1]{\textcolor[rgb]{0.56,0.35,0.01}{\textit{#1}}}
\newcommand{\AttributeTok}[1]{\textcolor[rgb]{0.77,0.63,0.00}{#1}}
\newcommand{\RegionMarkerTok}[1]{#1}
\newcommand{\InformationTok}[1]{\textcolor[rgb]{0.56,0.35,0.01}{\textbf{\textit{#1}}}}
\newcommand{\WarningTok}[1]{\textcolor[rgb]{0.56,0.35,0.01}{\textbf{\textit{#1}}}}
\newcommand{\AlertTok}[1]{\textcolor[rgb]{0.94,0.16,0.16}{#1}}
\newcommand{\ErrorTok}[1]{\textcolor[rgb]{0.64,0.00,0.00}{\textbf{#1}}}
\newcommand{\NormalTok}[1]{#1}
\usepackage{graphicx,grffile}
\makeatletter
\def\maxwidth{\ifdim\Gin@nat@width>\linewidth\linewidth\else\Gin@nat@width\fi}
\def\maxheight{\ifdim\Gin@nat@height>\textheight\textheight\else\Gin@nat@height\fi}
\makeatother
% Scale images if necessary, so that they will not overflow the page
% margins by default, and it is still possible to overwrite the defaults
% using explicit options in \includegraphics[width, height, ...]{}
\setkeys{Gin}{width=\maxwidth,height=\maxheight,keepaspectratio}
\IfFileExists{parskip.sty}{%
\usepackage{parskip}
}{% else
\setlength{\parindent}{0pt}
\setlength{\parskip}{6pt plus 2pt minus 1pt}
}
\setlength{\emergencystretch}{3em}  % prevent overfull lines
\providecommand{\tightlist}{%
  \setlength{\itemsep}{0pt}\setlength{\parskip}{0pt}}
\setcounter{secnumdepth}{0}
% Redefines (sub)paragraphs to behave more like sections
\ifx\paragraph\undefined\else
\let\oldparagraph\paragraph
\renewcommand{\paragraph}[1]{\oldparagraph{#1}\mbox{}}
\fi
\ifx\subparagraph\undefined\else
\let\oldsubparagraph\subparagraph
\renewcommand{\subparagraph}[1]{\oldsubparagraph{#1}\mbox{}}
\fi

%%% Use protect on footnotes to avoid problems with footnotes in titles
\let\rmarkdownfootnote\footnote%
\def\footnote{\protect\rmarkdownfootnote}

%%% Change title format to be more compact
\usepackage{titling}

% Create subtitle command for use in maketitle
\providecommand{\subtitle}[1]{
  \posttitle{
    \begin{center}\large#1\end{center}
    }
}

\setlength{\droptitle}{-2em}

  \title{LinearModelReporting.R}
    \pretitle{\vspace{\droptitle}\centering\huge}
  \posttitle{\par}
    \author{GlennTattersall}
    \preauthor{\centering\large\emph}
  \postauthor{\par}
      \predate{\centering\large\emph}
  \postdate{\par}
    \date{2019-04-22}


\begin{document}
\maketitle

\begin{Shaded}
\begin{Highlighting}[]
\CommentTok{# Summary of Regression / Linear Models as HTML Table}

\CommentTok{# Source: https://strengejacke.github.io/sjPlot/articles/tab_model_estimates.html#a-simple-html-table-from-regression-results}

\CommentTok{# tab_model() is the pendant to plot_model(), however, instead of creating }
\CommentTok{# plots, tab_model() creates HTML-tables that will be displayed either in your }
\CommentTok{# IDE's viewer-pane, in a web browser or in a knitr-markdown-document }

\CommentTok{# HTML is the only output-format, you can't (directly) create a LaTex or PDF }
\CommentTok{# output from tab_model() and related table-functions. However, it is possible}
\CommentTok{# to easily export the tables into Microsoft Word or Libre Office Writer.}

\CommentTok{# This vignette shows how to create table from regression models with tab_model(). }

\CommentTok{# Note. Due to the custom CSS, the layout of the table inside a}
\CommentTok{# knitr-document differs from the output in the viewer-pane and web browser.}

\CommentTok{# Install packages in this order:}
\CommentTok{# sjlabelled -> sjmisc -> sjstats -> ggeffects -> sjPlot}

\CommentTok{# load packages}
\KeywordTok{library}\NormalTok{(sjPlot)}
\end{Highlighting}
\end{Shaded}

\begin{verbatim}
## Learn more about sjPlot with 'browseVignettes("sjPlot")'.
\end{verbatim}

\begin{Shaded}
\begin{Highlighting}[]
\KeywordTok{library}\NormalTok{(sjmisc)}
\KeywordTok{library}\NormalTok{(sjlabelled)}

\NormalTok{## sample data}
\KeywordTok{data}\NormalTok{(efc)}
\KeywordTok{str}\NormalTok{(efc)}
\end{Highlighting}
\end{Shaded}

\begin{verbatim}
## 'data.frame':    908 obs. of  26 variables:
##  $ c12hour : num  16 148 70 168 168 16 161 110 28 40 ...
##   ..- attr(*, "label")= chr "average number of hours of care per week"
##  $ e15relat: num  2 2 1 1 2 2 1 4 2 2 ...
##   ..- attr(*, "label")= chr "relationship to elder"
##   ..- attr(*, "labels")= Named num  1 2 3 4 5 6 7 8
##   .. ..- attr(*, "names")= chr  "spouse/partner" "child" "sibling" "daughter or son -in-law" ...
##  $ e16sex  : num  2 2 2 2 2 2 1 2 2 2 ...
##   ..- attr(*, "label")= chr "elder's gender"
##   ..- attr(*, "labels")= Named num  1 2
##   .. ..- attr(*, "names")= chr  "male" "female"
##  $ e17age  : num  83 88 82 67 84 85 74 87 79 83 ...
##   ..- attr(*, "label")= chr "elder' age"
##  $ e42dep  : num  3 3 3 4 4 4 4 4 4 4 ...
##   ..- attr(*, "label")= chr "elder's dependency"
##   ..- attr(*, "labels")= Named num  1 2 3 4
##   .. ..- attr(*, "names")= chr  "independent" "slightly dependent" "moderately dependent" "severely dependent"
##  $ c82cop1 : num  3 3 2 4 3 2 4 3 3 3 ...
##   ..- attr(*, "label")= chr "do you feel you cope well as caregiver?"
##   ..- attr(*, "labels")= Named num  1 2 3 4
##   .. ..- attr(*, "names")= chr  "never" "sometimes" "often" "always"
##  $ c83cop2 : num  2 3 2 1 2 2 2 2 2 2 ...
##   ..- attr(*, "label")= chr "do you find caregiving too demanding?"
##   ..- attr(*, "labels")= Named num  1 2 3 4
##   .. ..- attr(*, "names")= chr  "Never" "Sometimes" "Often" "Always"
##  $ c84cop3 : num  2 3 1 3 1 3 4 2 3 1 ...
##   ..- attr(*, "label")= chr "does caregiving cause difficulties in your relationship with your friends?"
##   ..- attr(*, "labels")= Named num  1 2 3 4
##   .. ..- attr(*, "names")= chr  "Never" "Sometimes" "Often" "Always"
##  $ c85cop4 : num  2 3 4 1 2 3 1 1 2 2 ...
##   ..- attr(*, "label")= chr "does caregiving have negative effect on your physical health?"
##   ..- attr(*, "labels")= Named num  1 2 3 4
##   .. ..- attr(*, "names")= chr  "Never" "Sometimes" "Often" "Always"
##  $ c86cop5 : num  1 4 1 1 2 3 1 1 2 1 ...
##   ..- attr(*, "label")= chr "does caregiving cause difficulties in your relationship with your family?"
##   ..- attr(*, "labels")= Named num  1 2 3 4
##   .. ..- attr(*, "names")= chr  "Never" "Sometimes" "Often" "Always"
##  $ c87cop6 : num  1 1 1 1 2 2 2 1 1 1 ...
##   ..- attr(*, "label")= chr "does caregiving cause financial difficulties?"
##   ..- attr(*, "labels")= Named num  1 2 3 4
##   .. ..- attr(*, "names")= chr  "Never" "Sometimes" "Often" "Always"
##  $ c88cop7 : num  2 3 1 1 1 2 4 2 3 1 ...
##   ..- attr(*, "label")= chr "do you feel trapped in your role as caregiver?"
##   ..- attr(*, "labels")= Named num  1 2 3 4
##   .. ..- attr(*, "names")= chr  "Never" "Sometimes" "Often" "Always"
##  $ c89cop8 : num  3 2 4 2 4 1 1 3 1 1 ...
##   ..- attr(*, "label")= chr "do you feel supported by friends/neighbours?"
##   ..- attr(*, "labels")= Named num  1 2 3 4
##   .. ..- attr(*, "names")= chr  "never" "sometimes" "often" "always"
##  $ c90cop9 : num  3 2 3 4 4 1 4 3 3 3 ...
##   ..- attr(*, "label")= chr "do you feel caregiving worthwhile?"
##   ..- attr(*, "labels")= Named num  1 2 3 4
##   .. ..- attr(*, "names")= chr  "never" "sometimes" "often" "always"
##  $ c160age : num  56 54 80 69 47 56 61 67 59 49 ...
##   ..- attr(*, "label")= chr "carer' age"
##  $ c161sex : num  2 2 1 1 2 1 2 2 2 2 ...
##   ..- attr(*, "label")= chr "carer's gender"
##   ..- attr(*, "labels")= Named num  1 2
##   .. ..- attr(*, "names")= chr  "Male" "Female"
##  $ c172code: num  2 2 1 2 2 2 2 2 NA 2 ...
##   ..- attr(*, "label")= chr "carer's level of education"
##   ..- attr(*, "labels")= Named num  1 2 3
##   .. ..- attr(*, "names")= chr  "low level of education" "intermediate level of education" "high level of education"
##  $ c175empl: num  1 1 0 0 0 1 0 0 0 0 ...
##   ..- attr(*, "label")= chr "are you currently employed?"
##   ..- attr(*, "labels")= Named num  0 1
##   .. ..- attr(*, "names")= chr  "no" "yes"
##  $ barthtot: num  75 75 35 0 25 60 5 35 15 0 ...
##   ..- attr(*, "label")= chr "Total score BARTHEL INDEX"
##  $ neg_c_7 : num  12 20 11 10 12 19 15 11 15 10 ...
##   ..- attr(*, "label")= chr "Negative impact with 7 items"
##  $ pos_v_4 : num  12 11 13 15 15 9 13 14 13 13 ...
##   ..- attr(*, "label")= chr "Positive value with 4 items"
##  $ quol_5  : num  14 10 7 12 19 8 20 20 8 15 ...
##   ..- attr(*, "label")= chr "Quality of life 5 items"
##  $ resttotn: num  0 4 0 2 2 1 0 0 0 1 ...
##   ..- attr(*, "label")= chr "Job restrictions"
##  $ tot_sc_e: num  4 0 1 0 1 3 0 1 2 1 ...
##   ..- attr(*, "label")= chr "Services for elderly"
##  $ n4pstu  : num  0 0 2 3 2 2 3 1 3 3 ...
##   ..- attr(*, "label")= chr "Care level"
##   ..- attr(*, "labels")= Named chr  "0" "1" "2" "3" ...
##   .. ..- attr(*, "names")= chr  "No Care Level" "Care Level 1" "Care Level 2" "Care Level 3" ...
##  $ nur_pst : num  NA NA 2 3 2 2 3 1 3 3 ...
##   ..- attr(*, "label")= chr "Care level"
##   ..- attr(*, "labels")= Named chr  "1" "2" "3"
##   .. ..- attr(*, "names")= chr  "Care Level 1" "Care Level 2" "Care Level 3/3+"
\end{verbatim}

\begin{Shaded}
\begin{Highlighting}[]
\KeywordTok{str}\NormalTok{(efc}\OperatorTok{$}\NormalTok{c12hour)}
\end{Highlighting}
\end{Shaded}

\begin{verbatim}
##  num [1:908] 16 148 70 168 168 16 161 110 28 40 ...
##  - attr(*, "label")= chr "average number of hours of care per week"
\end{verbatim}

\begin{Shaded}
\begin{Highlighting}[]
\CommentTok{# each numeric parameter also has a label attached to it}
\CommentTok{# see the Hmisc package:}
\NormalTok{Hmisc}\OperatorTok{::}\KeywordTok{label}\NormalTok{(efc}\OperatorTok{$}\NormalTok{c12hour)}
\end{Highlighting}
\end{Shaded}

\begin{verbatim}
## [1] "average number of hours of care per week"
\end{verbatim}

\begin{Shaded}
\begin{Highlighting}[]
\NormalTok{efc <-}\StringTok{ }\KeywordTok{as_factor}\NormalTok{(efc, c161sex, c172code)}

\CommentTok{# A simple HTML table from regression results}
\CommentTok{# First, we fit two linear models to demonstrate the tab_model()-function.}

\NormalTok{m1 <-}\StringTok{ }\KeywordTok{lm}\NormalTok{(barthtot }\OperatorTok{~}\StringTok{ }\NormalTok{c160age }\OperatorTok{+}\StringTok{ }\NormalTok{c12hour }\OperatorTok{+}\StringTok{ }\NormalTok{c161sex }\OperatorTok{+}\StringTok{ }\NormalTok{c172code, }\DataTypeTok{data =}\NormalTok{ efc)}
\NormalTok{m2 <-}\StringTok{ }\KeywordTok{lm}\NormalTok{(neg_c_}\DecValTok{7} \OperatorTok{~}\StringTok{ }\NormalTok{c160age }\OperatorTok{+}\StringTok{ }\NormalTok{c12hour }\OperatorTok{+}\StringTok{ }\NormalTok{c161sex }\OperatorTok{+}\StringTok{ }\NormalTok{e17age, }\DataTypeTok{data =}\NormalTok{ efc)}

\CommentTok{# The simplest way of producing the table output is by passing the fitted model}
\CommentTok{# as parameter. By default, estimates, confidence intervals (CI) and p-values }
\CommentTok{# (p) are reported. As summary, the numbers of observations as well as the }
\CommentTok{# R-squared values are shown.}

\KeywordTok{summary}\NormalTok{(m1)}
\end{Highlighting}
\end{Shaded}

\begin{verbatim}
## 
## Call:
## lm(formula = barthtot ~ c160age + c12hour + c161sex + c172code, 
##     data = efc)
## 
## Residuals:
##     Min      1Q  Median      3Q     Max 
## -75.144 -14.944   4.401  18.661  72.393 
## 
## Coefficients:
##             Estimate Std. Error t value Pr(>|t|)    
## (Intercept) 87.14994    4.68009  18.621  < 2e-16 ***
## c160age     -0.20716    0.07211  -2.873  0.00418 ** 
## c12hour     -0.27883    0.01865 -14.950  < 2e-16 ***
## c161sex2    -0.39402    2.08893  -0.189  0.85044    
## c172code2    1.36596    2.28440   0.598  0.55004    
## c172code3   -1.64045    2.84037  -0.578  0.56373    
## ---
## Signif. codes:  0 '***' 0.001 '**' 0.01 '*' 0.05 '.' 0.1 ' ' 1
## 
## Residual standard error: 25.35 on 815 degrees of freedom
##   (87 observations deleted due to missingness)
## Multiple R-squared:  0.2708, Adjusted R-squared:  0.2664 
## F-statistic: 60.54 on 5 and 815 DF,  p-value: < 2.2e-16
\end{verbatim}

\begin{Shaded}
\begin{Highlighting}[]
\CommentTok{# compare summary to tab_model:}
\KeywordTok{tab_model}\NormalTok{(m1)}
\end{Highlighting}
\end{Shaded}

~

Total score BARTHEL INDEX

Predictors

Estimates

CI

p

(Intercept)

87.15

77.98~--~96.32

\textless{}0.001

carer'age

-0.21

-0.35~--~-0.07

0.004

average number of hoursof care per week

-0.28

-0.32~--~-0.24

\textless{}0.001

Female

-0.39

-4.49~--~3.70

0.850

intermediate level ofeducation

1.37

-3.11~--~5.84

0.550

high level of education

-1.64

-7.21~--~3.93

0.564

Observations

821

R2 / adjusted R2

0.271 / 0.266

\begin{Shaded}
\begin{Highlighting}[]
\CommentTok{# Automatic labelling}
\KeywordTok{colnames}\NormalTok{(efc)}
\end{Highlighting}
\end{Shaded}

\begin{verbatim}
##  [1] "c12hour"  "e15relat" "e16sex"   "e17age"   "e42dep"   "c82cop1" 
##  [7] "c83cop2"  "c84cop3"  "c85cop4"  "c86cop5"  "c87cop6"  "c88cop7" 
## [13] "c89cop8"  "c90cop9"  "c160age"  "c161sex"  "c172code" "c175empl"
## [19] "barthtot" "neg_c_7"  "pos_v_4"  "quol_5"   "resttotn" "tot_sc_e"
## [25] "n4pstu"   "nur_pst"
\end{verbatim}

\begin{Shaded}
\begin{Highlighting}[]
\CommentTok{# columns look like quite unremarkable features, but look closely:}
\KeywordTok{str}\NormalTok{(efc}\OperatorTok{$}\NormalTok{c160age)}
\end{Highlighting}
\end{Shaded}

\begin{verbatim}
##  num [1:908] 56 54 80 69 47 56 61 67 59 49 ...
##  - attr(*, "label")= chr "carer' age"
\end{verbatim}

\begin{Shaded}
\begin{Highlighting}[]
\KeywordTok{str}\NormalTok{(efc}\OperatorTok{$}\NormalTok{c12hour)}
\end{Highlighting}
\end{Shaded}

\begin{verbatim}
##  num [1:908] 16 148 70 168 168 16 161 110 28 40 ...
##  - attr(*, "label")= chr "average number of hours of care per week"
\end{verbatim}

\begin{Shaded}
\begin{Highlighting}[]
\CommentTok{# As the sjPlot-packages features labelled data, the coefficients in the table}
\CommentTok{# are already labelled in this example. The name of the dependent variable(s) }
\CommentTok{# is used as main column header for each model. For non-labelled data, the }
\CommentTok{# coefficient names are shown.}

\CommentTok{# Turn off automatic labelling}
\CommentTok{# To turn off automatic labelling, use auto.label = FALSE, or provide an empty}
\CommentTok{# character vector for pred.labels and dv.labels.}

\KeywordTok{tab_model}\NormalTok{(m1, }\DataTypeTok{auto.label =} \OtherTok{FALSE}\NormalTok{)}
\end{Highlighting}
\end{Shaded}

~

barthtot

Predictors

Estimates

CI

p

(Intercept)

87.15

77.98~--~96.32

\textless{}0.001

c160age

-0.21

-0.35~--~-0.07

0.004

c12hour

-0.28

-0.32~--~-0.24

\textless{}0.001

c161sex2

-0.39

-4.49~--~3.70

0.850

c172code2

1.37

-3.11~--~5.84

0.550

c172code3

-1.64

-7.21~--~3.93

0.564

Observations

821

R2 / adjusted R2

0.271 / 0.266

\begin{Shaded}
\begin{Highlighting}[]
\CommentTok{# some categorical data are already sufficient}
\KeywordTok{data}\NormalTok{(mtcars)}
\NormalTok{m.mtcars <-}\StringTok{ }\KeywordTok{lm}\NormalTok{(mpg }\OperatorTok{~}\StringTok{ }\NormalTok{cyl }\OperatorTok{+}\StringTok{ }\NormalTok{hp }\OperatorTok{+}\StringTok{ }\NormalTok{wt, }\DataTypeTok{data =}\NormalTok{ mtcars)}
\KeywordTok{tab_model}\NormalTok{(m.mtcars)}
\end{Highlighting}
\end{Shaded}

~

mpg

Predictors

Estimates

CI

p

(Intercept)

38.75

35.25~--~42.25

\textless{}0.001

cyl

-0.94

-2.02~--~0.14

0.098

hp

-0.02

-0.04~--~0.01

0.140

wt

-3.17

-4.62~--~-1.72

\textless{}0.001

Observations

32

R2 / adjusted R2

0.843 / 0.826

\begin{Shaded}
\begin{Highlighting}[]
\CommentTok{# but maybe you want to add details, you can do so manually. Note you need to }
\CommentTok{# specify the intercept predictor as well in a linear model:}
\KeywordTok{tab_model}\NormalTok{(m.mtcars,  }
          \DataTypeTok{pred.labels=}\KeywordTok{c}\NormalTok{(}\StringTok{"(Intercept)"}\NormalTok{, }\StringTok{"Cylinders"}\NormalTok{, }\StringTok{"Horse Power"}\NormalTok{, }\StringTok{"Weight"}\NormalTok{))}
\end{Highlighting}
\end{Shaded}

~

mpg

Predictors

Estimates

CI

p

(Intercept)

38.75

35.25~--~42.25

\textless{}0.001

Cylinders

-0.94

-2.02~--~0.14

0.098

Horse Power

-0.02

-0.04~--~0.01

0.140

Weight

-3.17

-4.62~--~-1.72

\textless{}0.001

Observations

32

R2 / adjusted R2

0.843 / 0.826

\begin{Shaded}
\begin{Highlighting}[]
\CommentTok{# What to do about model intercept?}
\CommentTok{# You can forcibly remove the intercept, at which point, the intercept effect}
\CommentTok{# simply becomes encapsulated into one of the main categorical variables.}
\NormalTok{m1.}\DecValTok{0}\NormalTok{ <-}\StringTok{ }\KeywordTok{lm}\NormalTok{(barthtot }\OperatorTok{~}\StringTok{ }\NormalTok{c160age }\OperatorTok{+}\StringTok{ }\NormalTok{c12hour }\OperatorTok{+}\StringTok{ }\NormalTok{c161sex }\OperatorTok{+}\StringTok{ }\NormalTok{c172code }\OperatorTok{-}\StringTok{ }\DecValTok{1}\NormalTok{, }\DataTypeTok{data =}\NormalTok{ efc)}
\KeywordTok{tab_model}\NormalTok{(m1)}
\end{Highlighting}
\end{Shaded}

~

Total score BARTHEL INDEX

Predictors

Estimates

CI

p

(Intercept)

87.15

77.98~--~96.32

\textless{}0.001

carer'age

-0.21

-0.35~--~-0.07

0.004

average number of hoursof care per week

-0.28

-0.32~--~-0.24

\textless{}0.001

Female

-0.39

-4.49~--~3.70

0.850

intermediate level ofeducation

1.37

-3.11~--~5.84

0.550

high level of education

-1.64

-7.21~--~3.93

0.564

Observations

821

R2 / adjusted R2

0.271 / 0.266

\begin{Shaded}
\begin{Highlighting}[]
\KeywordTok{tab_model}\NormalTok{(m1.}\DecValTok{0}\NormalTok{)}
\end{Highlighting}
\end{Shaded}

~

Total score BARTHEL INDEX

Predictors

Estimates

CI

p

carer'age

-0.21

-0.35~--~-0.07

0.004

average number of hoursof care per week

-0.28

-0.32~--~-0.24

\textless{}0.001

Male

87.15

77.98~--~96.32

\textless{}0.001

Female

86.76

78.00~--~95.51

\textless{}0.001

intermediate level ofeducation

1.37

-3.11~--~5.84

0.550

high level of education

-1.64

-7.21~--~3.93

0.564

Observations

821

R2 / adjusted R2

0.874 / 0.873

\begin{Shaded}
\begin{Highlighting}[]
\CommentTok{# More than one model}
\CommentTok{# tab_model() can print multiple models at once, which are then printed }
\CommentTok{# side-by-side. Identical predictor coefficients are matched in a row.}
\KeywordTok{tab_model}\NormalTok{(m1, m2)}
\end{Highlighting}
\end{Shaded}

~

Total score BARTHEL INDEX

Negative impact with 7items

Predictors

Estimates

CI

p

Estimates

CI

p

(Intercept)

87.15

77.98~--~96.32

\textless{}0.001

9.83

7.34~--~12.33

\textless{}0.001

carer'age

-0.21

-0.35~--~-0.07

0.004

0.01

-0.01~--~0.03

0.359

average number of hoursof care per week

-0.28

-0.32~--~-0.24

\textless{}0.001

0.02

0.01~--~0.02

\textless{}0.001

Female

-0.39

-4.49~--~3.70

0.850

0.43

-0.15~--~1.01

0.147

intermediate level ofeducation

1.37

-3.11~--~5.84

0.550

high level of education

-1.64

-7.21~--~3.93

0.564

elder'age

0.01

-0.03~--~0.04

0.741

Observations

821

879

R2 / adjusted R2

0.271 / 0.266

0.067 / 0.063

\begin{Shaded}
\begin{Highlighting}[]
\CommentTok{# Generalized linear models}
\CommentTok{# For generalized linear models, the ouput is slightly adapted. }
\CommentTok{# Instead of Estimates, the column is named Odds Ratios, Incidence Rate Ratios}
\CommentTok{# etc., depending on the model. }
\CommentTok{# The coefficients are, by default, automatically}
\CommentTok{# converted (exponentiated). Furthermore, pseudo R-squared statistics are }
\CommentTok{# shown in the summary.}

\NormalTok{m3 <-}\StringTok{ }\KeywordTok{glm}\NormalTok{(}
\NormalTok{  tot_sc_e }\OperatorTok{~}\StringTok{ }\NormalTok{c160age }\OperatorTok{+}\StringTok{ }\NormalTok{c12hour }\OperatorTok{+}\StringTok{ }\NormalTok{c161sex }\OperatorTok{+}\StringTok{ }\NormalTok{c172code, }
  \DataTypeTok{data =}\NormalTok{ efc, }\DataTypeTok{family =} \KeywordTok{poisson}\NormalTok{(}\DataTypeTok{link =} \StringTok{"log"}\NormalTok{)}
\NormalTok{)}

\NormalTok{efc}\OperatorTok{$}\NormalTok{neg_c_7d <-}\StringTok{ }\KeywordTok{ifelse}\NormalTok{(efc}\OperatorTok{$}\NormalTok{neg_c_}\DecValTok{7} \OperatorTok{<}\StringTok{ }\KeywordTok{median}\NormalTok{(efc}\OperatorTok{$}\NormalTok{neg_c_}\DecValTok{7}\NormalTok{, }\DataTypeTok{na.rm =} \OtherTok{TRUE}\NormalTok{), }\DecValTok{0}\NormalTok{, }\DecValTok{1}\NormalTok{)}

\NormalTok{m4 <-}\StringTok{ }\KeywordTok{glm}\NormalTok{(}
\NormalTok{  neg_c_7d }\OperatorTok{~}\StringTok{ }\NormalTok{c161sex }\OperatorTok{+}\StringTok{ }\NormalTok{barthtot }\OperatorTok{+}\StringTok{ }\NormalTok{c172code,}
  \DataTypeTok{data =}\NormalTok{ efc, }\DataTypeTok{family =} \KeywordTok{binomial}\NormalTok{(}\DataTypeTok{link =} \StringTok{"logit"}\NormalTok{)}
\NormalTok{)}

\KeywordTok{tab_model}\NormalTok{(m3, m4)}
\end{Highlighting}
\end{Shaded}

~

Services for elderly

neg c 7 d

Predictors

Incidence Rate Ratios

CI

p

Odds Ratios

CI

p

(Intercept)

0.30

0.21~--~0.45

\textless{}0.001

6.54

3.62~--~11.81

\textless{}0.001

carer'age

1.01

1.01~--~1.02

\textless{}0.001

average number of hoursof care per week

1.00

1.00~--~1.00

\textless{}0.001

Female

1.01

0.86~--~1.19

0.867

1.87

1.30~--~2.68

0.001

intermediate level ofeducation

1.47

1.21~--~1.78

\textless{}0.001

1.23

0.84~--~1.82

0.288

high level of education

1.90

1.52~--~2.37

\textless{}0.001

1.37

0.84~--~2.23

0.204

Total score BARTHEL INDEX

0.97

0.96~--~0.97

\textless{}0.001

Observations

840

815

Cox \& Snell's R2 / Nagelkerke's R2

0.083 / 0.106

0.184 / 0.247

\begin{Shaded}
\begin{Highlighting}[]
\CommentTok{# Untransformed estimates on the linear scale}
\CommentTok{# To plot the estimates on the linear scale, use transform = NULL.}
\KeywordTok{tab_model}\NormalTok{(m3, m4, }\DataTypeTok{transform =} \OtherTok{NULL}\NormalTok{, }\DataTypeTok{auto.label =}\NormalTok{ T)}
\end{Highlighting}
\end{Shaded}

~

Services for elderly

neg c 7 d

Predictors

Log-Mean

CI

p

Log-Odds

CI

p

(Intercept)

-1.19

-1.58~--~-0.80

\textless{}0.001

1.88

1.29~--~2.47

\textless{}0.001

carer'age

0.01

0.01~--~0.02

\textless{}0.001

average number of hoursof care per week

0.00

0.00~--~0.00

\textless{}0.001

Female

0.01

-0.15~--~0.17

0.867

0.63

0.26~--~0.99

0.001

intermediate level ofeducation

0.39

0.19~--~0.58

\textless{}0.001

0.21

-0.18~--~0.60

0.288

high level of education

0.64

0.42~--~0.86

\textless{}0.001

0.31

-0.17~--~0.80

0.204

Total score BARTHEL INDEX

-0.03

-0.04~--~-0.03

\textless{}0.001

Observations

840

815

Cox \& Snell's R2 / Nagelkerke's R2

0.083 / 0.106

0.184 / 0.247

\begin{Shaded}
\begin{Highlighting}[]
\CommentTok{# More complex models}
\CommentTok{# Other models, like hurdle- or zero-inflated models, also work with tab_model().}
\CommentTok{# In this case, the zero inflation model is indicated in the table.}
\CommentTok{# Use show.zeroinf = FALSE to hide this part from the table.}

\KeywordTok{library}\NormalTok{(pscl)}
\end{Highlighting}
\end{Shaded}

\begin{verbatim}
## Classes and Methods for R developed in the
## Political Science Computational Laboratory
## Department of Political Science
## Stanford University
## Simon Jackman
## hurdle and zeroinfl functions by Achim Zeileis
\end{verbatim}

\begin{Shaded}
\begin{Highlighting}[]
\KeywordTok{data}\NormalTok{(bioChemists)}

\NormalTok{m5 <-}\StringTok{ }\KeywordTok{zeroinfl}\NormalTok{(art }\OperatorTok{~}\StringTok{ }\NormalTok{. }\OperatorTok{|}\StringTok{ }\NormalTok{., }\DataTypeTok{data =}\NormalTok{ bioChemists)}
\KeywordTok{tab_model}\NormalTok{(m5)}
\end{Highlighting}
\end{Shaded}

~

Dependent variable

Predictors

Incidence Rate Ratios

CI

p

(Intercept)

1.90

1.50~--~2.41

\textless{}0.001

femWomen

0.81

0.72~--~0.92

0.001

marMarried

1.11

0.97~--~1.28

0.145

kid5

0.87

0.79~--~0.95

0.003

phd

0.99

0.94~--~1.06

0.842

ment

1.02

1.01~--~1.02

\textless{}0.001

Zero-Inflated Model

(Intercept)

0.56

0.21~--~1.52

0.257

femWomen

1.12

0.64~--~1.93

0.695

marMarried

0.70

0.38~--~1.31

0.265

kid5

1.24

0.85~--~1.83

0.269

phd

1.00

0.75~--~1.33

0.993

ment

0.87

0.80~--~0.96

0.003

\begin{Shaded}
\begin{Highlighting}[]
\KeywordTok{tab_model}\NormalTok{(m5, }\DataTypeTok{show.zeroinf =}\NormalTok{ F)}
\end{Highlighting}
\end{Shaded}

~

Dependent variable

Predictors

Incidence Rate Ratios

CI

p

(Intercept)

1.90

1.50~--~2.41

\textless{}0.001

femWomen

0.81

0.72~--~0.92

0.001

marMarried

1.11

0.97~--~1.28

0.145

kid5

0.87

0.79~--~0.95

0.003

phd

0.99

0.94~--~1.06

0.842

ment

1.02

1.01~--~1.02

\textless{}0.001

\begin{Shaded}
\begin{Highlighting}[]
\CommentTok{# You can combine any model in one table.}

\KeywordTok{tab_model}\NormalTok{(m1, m3, }\DataTypeTok{auto.label =} \OtherTok{FALSE}\NormalTok{)}
\end{Highlighting}
\end{Shaded}

~

barthtot

tot\_sc\_e

Predictors

Estimates

CI

p

Incidence Rate Ratios

CI

p

(Intercept)

87.15

77.98~--~96.32

\textless{}0.001

0.30

0.21~--~0.45

\textless{}0.001

c160age

-0.21

-0.35~--~-0.07

0.004

1.01

1.01~--~1.02

\textless{}0.001

c12hour

-0.28

-0.32~--~-0.24

\textless{}0.001

1.00

1.00~--~1.00

\textless{}0.001

c161sex2

-0.39

-4.49~--~3.70

0.850

1.01

0.86~--~1.19

0.867

c172code2

1.37

-3.11~--~5.84

0.550

1.47

1.21~--~1.78

\textless{}0.001

c172code3

-1.64

-7.21~--~3.93

0.564

1.90

1.52~--~2.37

\textless{}0.001

Observations

821

840

R2 / adjusted R2

0.271 / 0.266

0.083 / 0.106

\begin{Shaded}
\begin{Highlighting}[]
\CommentTok{# Show or hide further columns}
\CommentTok{# tab_model() has some argument that allow to show or hide specific columns }
\CommentTok{# from the output:}

\CommentTok{# show.est to show/hide the column with model estimates.}
\KeywordTok{tab_model}\NormalTok{(m1, m3, }\DataTypeTok{auto.label =} \OtherTok{FALSE}\NormalTok{,  }\DataTypeTok{show.est=}\OtherTok{FALSE}\NormalTok{)}
\end{Highlighting}
\end{Shaded}

~

barthtot

tot\_sc\_e

Predictors

p

p

(Intercept)

\textless{}0.001

\textless{}0.001

c160age

0.004

\textless{}0.001

c12hour

\textless{}0.001

\textless{}0.001

c161sex2

0.850

0.867

c172code2

0.550

\textless{}0.001

c172code3

0.564

\textless{}0.001

Observations

821

840

R2 / adjusted R2

0.271 / 0.266

0.083 / 0.106

\begin{Shaded}
\begin{Highlighting}[]
\CommentTok{# show.ci to show/hide the column with confidence intervals.}
\KeywordTok{tab_model}\NormalTok{(m1, m3, }\DataTypeTok{auto.label =} \OtherTok{FALSE}\NormalTok{, }\DataTypeTok{show.ci=}\OtherTok{FALSE}\NormalTok{)}
\end{Highlighting}
\end{Shaded}

~

barthtot

tot\_sc\_e

Predictors

Estimates

p

Incidence Rate Ratios

p

(Intercept)

87.15

\textless{}0.001

0.30

\textless{}0.001

c160age

-0.21

0.004

1.01

\textless{}0.001

c12hour

-0.28

\textless{}0.001

1.00

\textless{}0.001

c161sex2

-0.39

0.850

1.01

0.867

c172code2

1.37

0.550

1.47

\textless{}0.001

c172code3

-1.64

0.564

1.90

\textless{}0.001

Observations

821

840

R2 / adjusted R2

0.271 / 0.266

0.083 / 0.106

\begin{Shaded}
\begin{Highlighting}[]
\CommentTok{# show.se to show/hide the column with standard errors.}
\KeywordTok{tab_model}\NormalTok{(m1, m3,  }\DataTypeTok{auto.label =} \OtherTok{FALSE}\NormalTok{, }\DataTypeTok{show.se=}\OtherTok{FALSE}\NormalTok{)}
\end{Highlighting}
\end{Shaded}

~

barthtot

tot\_sc\_e

Predictors

Estimates

CI

p

Incidence Rate Ratios

CI

p

(Intercept)

87.15

77.98~--~96.32

\textless{}0.001

0.30

0.21~--~0.45

\textless{}0.001

c160age

-0.21

-0.35~--~-0.07

0.004

1.01

1.01~--~1.02

\textless{}0.001

c12hour

-0.28

-0.32~--~-0.24

\textless{}0.001

1.00

1.00~--~1.00

\textless{}0.001

c161sex2

-0.39

-4.49~--~3.70

0.850

1.01

0.86~--~1.19

0.867

c172code2

1.37

-3.11~--~5.84

0.550

1.47

1.21~--~1.78

\textless{}0.001

c172code3

-1.64

-7.21~--~3.93

0.564

1.90

1.52~--~2.37

\textless{}0.001

Observations

821

840

R2 / adjusted R2

0.271 / 0.266

0.083 / 0.106

\begin{Shaded}
\begin{Highlighting}[]
\CommentTok{# show.std to show/hide the column with standardized estimates }
\CommentTok{# (and their standard errors).}
\KeywordTok{tab_model}\NormalTok{(m1, m3, }\DataTypeTok{auto.label =} \OtherTok{FALSE}\NormalTok{, }\DataTypeTok{show.std=}\NormalTok{T, }\DataTypeTok{show.ci=}\NormalTok{F)}
\end{Highlighting}
\end{Shaded}

~

barthtot

tot\_sc\_e

Predictors

Estimates

std. Beta

p

Incidence Rate Ratios

p

(Intercept)

87.15

\textless{}0.001

0.30

\textless{}0.001

c160age

-0.21

-0.09

0.004

1.01

\textless{}0.001

c12hour

-0.28

-0.48

\textless{}0.001

1.00

\textless{}0.001

c161sex2

-0.39

-0.01

0.850

1.01

0.867

c172code2

1.37

0.02

0.550

1.47

\textless{}0.001

c172code3

-1.64

-0.02

0.564

1.90

\textless{}0.001

Observations

821

840

R2 / adjusted R2

0.271 / 0.266

0.083 / 0.106

\begin{Shaded}
\begin{Highlighting}[]
\CommentTok{# show.p to show/hide the column with p-values.}
\KeywordTok{tab_model}\NormalTok{(m1, m3, }\DataTypeTok{auto.label =} \OtherTok{FALSE}\NormalTok{, }\DataTypeTok{show.p=}\OtherTok{FALSE}\NormalTok{, }\DataTypeTok{show.ci=}\NormalTok{F)}
\end{Highlighting}
\end{Shaded}

~

barthtot

tot\_sc\_e

Predictors

Estimates

Incidence Rate Ratios

(Intercept)

87.15

0.30

c160age

-0.21

1.01

c12hour

-0.28

1.00

c161sex2

-0.39

1.01

c172code2

1.37

1.47

c172code3

-1.64

1.90

Observations

821

840

R2 / adjusted R2

0.271 / 0.266

0.083 / 0.106

\begin{Shaded}
\begin{Highlighting}[]
\CommentTok{# show.stat to show/hide the column with the coefficients' test statistics.}
\KeywordTok{tab_model}\NormalTok{(m1, m3, }\DataTypeTok{auto.label =} \OtherTok{FALSE}\NormalTok{, }\DataTypeTok{show.stat=}\NormalTok{T, }\DataTypeTok{show.ci=}\NormalTok{F)}
\end{Highlighting}
\end{Shaded}

~

barthtot

tot\_sc\_e

Predictors

Estimates

Statistic

p

Incidence Rate Ratios

Statistic

p

(Intercept)

87.15

18.62

\textless{}0.001

0.30

-5.97

\textless{}0.001

c160age

-0.21

-2.87

0.004

1.01

4.41

\textless{}0.001

c12hour

-0.28

-14.95

\textless{}0.001

1.00

3.72

\textless{}0.001

c161sex2

-0.39

-0.19

0.850

1.01

0.17

0.867

c172code2

1.37

0.60

0.550

1.47

3.89

\textless{}0.001

c172code3

-1.64

-0.58

0.564

1.90

5.65

\textless{}0.001

Observations

821

840

R2 / adjusted R2

0.271 / 0.266

0.083 / 0.106

\begin{Shaded}
\begin{Highlighting}[]
\KeywordTok{tab_model}\NormalTok{(m1, m3, }\DataTypeTok{auto.label =} \OtherTok{FALSE}\NormalTok{, }\DataTypeTok{show.stat=}\NormalTok{F, }\DataTypeTok{show.ci=}\NormalTok{F)}
\end{Highlighting}
\end{Shaded}

~

barthtot

tot\_sc\_e

Predictors

Estimates

p

Incidence Rate Ratios

p

(Intercept)

87.15

\textless{}0.001

0.30

\textless{}0.001

c160age

-0.21

0.004

1.01

\textless{}0.001

c12hour

-0.28

\textless{}0.001

1.00

\textless{}0.001

c161sex2

-0.39

0.850

1.01

0.867

c172code2

1.37

0.550

1.47

\textless{}0.001

c172code3

-1.64

0.564

1.90

\textless{}0.001

Observations

821

840

R2 / adjusted R2

0.271 / 0.266

0.083 / 0.106

\begin{Shaded}
\begin{Highlighting}[]
\CommentTok{# show.df for linear mixed models, when p-values are based on degrees of}
\CommentTok{# freedom with Kenward-Rogers approximation, these degrees of freedom are shown.}
\CommentTok{# p.val needs to be set to "kr"}

\KeywordTok{library}\NormalTok{(lme4)}
\end{Highlighting}
\end{Shaded}

\begin{verbatim}
## Loading required package: Matrix
\end{verbatim}

\begin{Shaded}
\begin{Highlighting}[]
\KeywordTok{data}\NormalTok{(sleepstudy)}
\KeywordTok{str}\NormalTok{(sleepstudy)}
\end{Highlighting}
\end{Shaded}

\begin{verbatim}
## 'data.frame':    180 obs. of  3 variables:
##  $ Reaction: num  250 259 251 321 357 ...
##  $ Days    : num  0 1 2 3 4 5 6 7 8 9 ...
##  $ Subject : Factor w/ 18 levels "308","309","310",..: 1 1 1 1 1 1 1 1 1 1 ...
\end{verbatim}

\begin{Shaded}
\begin{Highlighting}[]
\NormalTok{me1<-}\KeywordTok{lmer}\NormalTok{(Reaction }\OperatorTok{~}\StringTok{ }\NormalTok{Days }\OperatorTok{+}\StringTok{ }\NormalTok{(}\DecValTok{1}\OperatorTok{|}\NormalTok{Subject), }\DataTypeTok{data=}\NormalTok{sleepstudy)}
\KeywordTok{tab_model}\NormalTok{(me1, }\DataTypeTok{auto.label =} \OtherTok{FALSE}\NormalTok{, }\DataTypeTok{show.stat=}\NormalTok{T, }\DataTypeTok{show.se=}\NormalTok{T, }\DataTypeTok{show.df=}\NormalTok{T, }
          \DataTypeTok{p.val=}\StringTok{"kr"}\NormalTok{)}
\end{Highlighting}
\end{Shaded}

~

Reaction

Predictors

Estimates

std. Error

CI

Statistic

p

df

(Intercept)

251.41

9.75

232.30~--~270.51

25.79

\textless{}0.001

23.00

Days

10.47

0.80

8.89~--~12.04

13.02

\textless{}0.001

161.00

Random Effects

σ2

960.46

τ00 Subject

1378.18

ICC Subject

0.59

Observations

180

Marginal R2 / Conditional R2

0.280 / 0.704

\begin{Shaded}
\begin{Highlighting}[]
\CommentTok{# Adding columns}
\CommentTok{# In the following example, standard errors, standardized coefficients }
\CommentTok{# and test statistics are also shown.}

\KeywordTok{tab_model}\NormalTok{(m1, }\DataTypeTok{show.se =} \OtherTok{TRUE}\NormalTok{, }\DataTypeTok{show.std =} \OtherTok{TRUE}\NormalTok{, }\DataTypeTok{show.stat =} \OtherTok{TRUE}\NormalTok{)}
\end{Highlighting}
\end{Shaded}

~

Total score BARTHEL INDEX

Predictors

Estimates

std. Error

std. Beta

standardized std. Error

CI

standardized CI

Statistic

p

(Intercept)

87.15

4.68

77.98~--~96.32

18.62

\textless{}0.001

carer'age

-0.21

0.07

-0.09

0.03

-0.35~--~-0.07

-0.16~--~-0.03

-2.87

0.004

average number of hoursof care per week

-0.28

0.02

-0.48

0.03

-0.32~--~-0.24

-0.54~--~-0.42

-14.95

\textless{}0.001

Female

-0.39

2.09

-0.01

0.03

-4.49~--~3.70

-0.06~--~0.05

-0.19

0.850

intermediate level ofeducation

1.37

2.28

0.02

0.04

-3.11~--~5.84

-0.05~--~0.10

0.60

0.550

high level of education

-1.64

2.84

-0.02

0.04

-7.21~--~3.93

-0.09~--~0.05

-0.58

0.564

Observations

821

R2 / adjusted R2

0.271 / 0.266

\begin{Shaded}
\begin{Highlighting}[]
\CommentTok{# Removing columns}
\CommentTok{# In the following example, default columns are removed.}

\KeywordTok{tab_model}\NormalTok{(m3, m4, }\DataTypeTok{show.ci =} \OtherTok{FALSE}\NormalTok{, }\DataTypeTok{show.p =} \OtherTok{FALSE}\NormalTok{, }\DataTypeTok{auto.label =} \OtherTok{FALSE}\NormalTok{)}
\end{Highlighting}
\end{Shaded}

~

tot\_sc\_e

neg\_c\_7d

Predictors

Incidence Rate Ratios

Odds Ratios

(Intercept)

0.30

6.54

c160age

1.01

c12hour

1.00

c161sex2

1.01

1.87

c172code2

1.47

1.23

c172code3

1.90

1.37

barthtot

0.97

Observations

840

815

Cox \& Snell's R2 / Nagelkerke's R2

0.083 / 0.106

0.184 / 0.247

\begin{Shaded}
\begin{Highlighting}[]
\CommentTok{# Removing and sorting columns}
\CommentTok{# Another way to remove columns, which also allows to reorder the columns, }
\CommentTok{# is the col.order-argument. This is a character vector, where each element}
\CommentTok{# indicates a column in the output. The value est, for instance, }
\CommentTok{# indicates the estimates, while std.est is the column for standardized}
\CommentTok{# estimates and so on.}

\CommentTok{# By default, col.order contains all possible columns. All columns that}
\CommentTok{# should shown (see previous tables, for example using show.se = TRUE to }
\CommentTok{# show standard errors, or show.st = TRUE to show standardized estimates) are }
\CommentTok{# then printed by default. Colums that are excluded from col.order are not }
\CommentTok{# shown, no matter if the show-arguments are TRUE or FALSE. }
\CommentTok{# So if show.se = TRUE, but col.order does not contain the element "se", }
\CommentTok{# standard errors are not shown. On the other hand, if show.est = FALSE,}
\CommentTok{# but col.order does include the element "est", the columns with estimates }
\CommentTok{# are not shown.}
\CommentTok{# In summary, col.order can be used to exclude columns from the table and }
\CommentTok{# to change the order of colums.}


\KeywordTok{tab_model}\NormalTok{(}
\NormalTok{  m1, }\DataTypeTok{show.se =} \OtherTok{TRUE}\NormalTok{, }\DataTypeTok{show.std =} \OtherTok{TRUE}\NormalTok{, }\DataTypeTok{show.stat =} \OtherTok{TRUE}\NormalTok{,}
  \DataTypeTok{col.order =} \KeywordTok{c}\NormalTok{(}\StringTok{"p"}\NormalTok{, }\StringTok{"stat"}\NormalTok{, }\StringTok{"est"}\NormalTok{, }\StringTok{"std.se"}\NormalTok{, }\StringTok{"se"}\NormalTok{, }\StringTok{"std.est"}\NormalTok{)}
\NormalTok{)}
\end{Highlighting}
\end{Shaded}

~

Total score BARTHEL INDEX

Predictors

p

Statistic

Estimates

standardized std. Error

std. Error

std. Beta

(Intercept)

\textless{}0.001

18.62

87.15

4.68

carer'age

0.004

-2.87

-0.21

0.03

0.07

-0.09

average number of hoursof care per week

\textless{}0.001

-14.95

-0.28

0.03

0.02

-0.48

Female

0.850

-0.19

-0.39

0.03

2.09

-0.01

intermediate level ofeducation

0.550

0.60

1.37

0.04

2.28

0.02

high level of education

0.564

-0.58

-1.64

0.04

2.84

-0.02

Observations

821

R2 / adjusted R2

0.271 / 0.266

\begin{Shaded}
\begin{Highlighting}[]
\CommentTok{# Collapsing columns}
\CommentTok{# With collapse.ci and collapse.se, the columns for confidence intervals }
\CommentTok{# and standard errors can be collapsed into one column together with the}
\CommentTok{# estimates. Sometimes this table layout is required.}

\KeywordTok{tab_model}\NormalTok{(m1, }\DataTypeTok{collapse.ci =} \OtherTok{TRUE}\NormalTok{)}
\end{Highlighting}
\end{Shaded}

~

Total score BARTHEL INDEX

Predictors

Estimates

p

(Intercept)

87.15(77.98~--~96.32)

\textless{}0.001

carer'age

-0.21(-0.35~--~-0.07)

0.004

average number of hoursof care per week

-0.28(-0.32~--~-0.24)

\textless{}0.001

Female

-0.39(-4.49~--~3.70)

0.850

intermediate level ofeducation

1.37(-3.11~--~5.84)

0.550

high level of education

-1.64(-7.21~--~3.93)

0.564

Observations

821

R2 / adjusted R2

0.271 / 0.266

\begin{Shaded}
\begin{Highlighting}[]
\CommentTok{# Defining own labels}
\CommentTok{# There are different options to change the labels of the column headers}
\CommentTok{# or coefficients, e.g. with:}

\CommentTok{# pred.labels to change the names of the coefficients in the Predictors column. }
\CommentTok{# Note that the length of pred.labels must exactly match the amount of predictors }
\CommentTok{# in the Predictor column.}
\CommentTok{# dv.labels to change the names of the model columns, which are labelled with }
\CommentTok{# the variable labels / names from the dependent variables.}
\CommentTok{# Furthermore, there are various string-arguments, to change the name of }
\CommentTok{# column headings.}

\KeywordTok{tab_model}\NormalTok{(}
\NormalTok{  m1, m2, }
  \DataTypeTok{pred.labels =} \KeywordTok{c}\NormalTok{(}\StringTok{"Intercept"}\NormalTok{, }\StringTok{"Age (Carer)"}\NormalTok{, }\StringTok{"Hours per Week"}\NormalTok{, }\StringTok{"Gender (Carer)"}\NormalTok{,}
                  \StringTok{"Education: middle (Carer)"}\NormalTok{, }\StringTok{"Education: high (Carer)"}\NormalTok{, }
                  \StringTok{"Age (Older Person)"}\NormalTok{),}
  \DataTypeTok{dv.labels =} \KeywordTok{c}\NormalTok{(}\StringTok{"First Model"}\NormalTok{, }\StringTok{"M2"}\NormalTok{),}
  \DataTypeTok{string.pred =} \StringTok{"Coefficient"}\NormalTok{,}
  \DataTypeTok{string.ci =} \StringTok{"Conf. Int (95%)"}\NormalTok{,}
  \DataTypeTok{string.p =} \StringTok{"P-Value"}
\NormalTok{)}
\end{Highlighting}
\end{Shaded}

~

First Model

M2

Coefficient

Estimates

Conf. Int (95\%)

P-Value

Estimates

Conf. Int (95\%)

P-Value

Intercept

87.15

77.98~--~96.32

\textless{}0.001

9.83

7.34~--~12.33

\textless{}0.001

Age (Carer)

-0.21

-0.35~--~-0.07

0.004

0.01

-0.01~--~0.03

0.359

Hours per Week

-0.28

-0.32~--~-0.24

\textless{}0.001

0.02

0.01~--~0.02

\textless{}0.001

Gender (Carer)

-0.39

-4.49~--~3.70

0.850

0.43

-0.15~--~1.01

0.147

Education: middle (Carer)

1.37

-3.11~--~5.84

0.550

Education: high (Carer)

-1.64

-7.21~--~3.93

0.564

Age (Older Person)

0.01

-0.03~--~0.04

0.741

Observations

821

879

R2 / adjusted R2

0.271 / 0.266

0.067 / 0.063

\begin{Shaded}
\begin{Highlighting}[]
\CommentTok{# I don't think there is a way to change the title of the "Estimates" column?}

\CommentTok{# First Model   M2}
\CommentTok{# Show asterisks instead of numeric p-values}
\CommentTok{# You can change the style of how p-values are displayed with the argument}
\CommentTok{# p.style. With p.style = "asterisk", the p-values are indicated as * in }
\CommentTok{# the table.}

\KeywordTok{tab_model}\NormalTok{(m1, m2, }\DataTypeTok{p.style =} \StringTok{"a"}\NormalTok{)}
\end{Highlighting}
\end{Shaded}

~

Total score BARTHEL INDEX

Negative impact with 7items

Predictors

Estimates

CI

Estimates

CI

(Intercept)

87.15 ***

77.98~--~96.32

9.83 ***

7.34~--~12.33

carer'age

-0.21 **

-0.35~--~-0.07

0.01

-0.01~--~0.03

average number of hoursof care per week

-0.28 ***

-0.32~--~-0.24

0.02 ***

0.01~--~0.02

Female

-0.39

-4.49~--~3.70

0.43

-0.15~--~1.01

intermediate level ofeducation

1.37

-3.11~--~5.84

high level of education

-1.64

-7.21~--~3.93

elder'age

0.01

-0.03~--~0.04

Observations

821

879

R2 / adjusted R2

0.271 / 0.266

0.067 / 0.063

\begin{itemize}
\tightlist
\item
  p\textless{}0.05~~~** p\textless{}0.01~~~*** p\textless{}0.001
\end{itemize}

\begin{Shaded}
\begin{Highlighting}[]
\CommentTok{# Note: I personally find this annoying as it does not show p values at all but}
\CommentTok{# gives an impression of importance that may not be warranted.  I.e. when }
\CommentTok{# do you normally care about the significance of the intercept term?  Or does}
\CommentTok{# your field really care about p values, so why use *** to inflate or guide}
\CommentTok{# the reader toward emphasising something that they should discern themselves.}

\CommentTok{# Automatic matching for named vectors}
\CommentTok{# Another way to easily assign labels are named vectors. In this case,}
\CommentTok{# it doesn't matter if pred.labels has more labels than coefficients in the }
\CommentTok{# model(s), or in which order the labels are passed to tab_model(). The only}
\CommentTok{# requirement is that the labels' names equal the coefficients names as they}
\CommentTok{# appear in the summary()-output.}

\CommentTok{# example, coefficients are "c161sex2" or "c172code3"}
\KeywordTok{summary}\NormalTok{(m1)}
\end{Highlighting}
\end{Shaded}

\begin{verbatim}
## 
## Call:
## lm(formula = barthtot ~ c160age + c12hour + c161sex + c172code, 
##     data = efc)
## 
## Residuals:
##     Min      1Q  Median      3Q     Max 
## -75.144 -14.944   4.401  18.661  72.393 
## 
## Coefficients:
##             Estimate Std. Error t value Pr(>|t|)    
## (Intercept) 87.14994    4.68009  18.621  < 2e-16 ***
## c160age     -0.20716    0.07211  -2.873  0.00418 ** 
## c12hour     -0.27883    0.01865 -14.950  < 2e-16 ***
## c161sex2    -0.39402    2.08893  -0.189  0.85044    
## c172code2    1.36596    2.28440   0.598  0.55004    
## c172code3   -1.64045    2.84037  -0.578  0.56373    
## ---
## Signif. codes:  0 '***' 0.001 '**' 0.01 '*' 0.05 '.' 0.1 ' ' 1
## 
## Residual standard error: 25.35 on 815 degrees of freedom
##   (87 observations deleted due to missingness)
## Multiple R-squared:  0.2708, Adjusted R-squared:  0.2664 
## F-statistic: 60.54 on 5 and 815 DF,  p-value: < 2.2e-16
\end{verbatim}

\begin{Shaded}
\begin{Highlighting}[]
\CommentTok{# create a named vector, pl:}
\NormalTok{pl <-}\StringTok{ }\KeywordTok{c}\NormalTok{(}
  \StringTok{`}\DataTypeTok{(Intercept)}\StringTok{`}\NormalTok{ =}\StringTok{ "Intercept"}\NormalTok{,}
  \DataTypeTok{e17age =} \StringTok{"Age (Older Person)"}\NormalTok{,}
  \DataTypeTok{c160age =} \StringTok{"Age (Carer)"}\NormalTok{,}
  \DataTypeTok{c12hour =} \StringTok{"Hours per Week"}\NormalTok{,}
  \DataTypeTok{barthtot =} \StringTok{"Barthel-Index"}\NormalTok{,}
  \DataTypeTok{c161sex2 =} \StringTok{"Gender (Carer)"}\NormalTok{,}
  \DataTypeTok{c172code2 =} \StringTok{"Education: middle (Carer)"}\NormalTok{,}
  \DataTypeTok{c172code3 =} \StringTok{"Education: high (Carer)"}\NormalTok{,}
  \DataTypeTok{a_non_used_label =} \StringTok{"We don't care"}
\NormalTok{)}

\KeywordTok{cbind}\NormalTok{(pl)}
\end{Highlighting}
\end{Shaded}

\begin{verbatim}
##                  pl                         
## (Intercept)      "Intercept"                
## e17age           "Age (Older Person)"       
## c160age          "Age (Carer)"              
## c12hour          "Hours per Week"           
## barthtot         "Barthel-Index"            
## c161sex2         "Gender (Carer)"           
## c172code2        "Education: middle (Carer)"
## c172code3        "Education: high (Carer)"  
## a_non_used_label "We don't care"
\end{verbatim}

\begin{Shaded}
\begin{Highlighting}[]
\CommentTok{# see how pl is actually named, so you can still use the column names in the}
\CommentTok{# model call but the pl variable holds more informative information that}
\CommentTok{# includes words, spaces, capital letters etc..}

\KeywordTok{tab_model}\NormalTok{(}
\NormalTok{  m1, m2, m3, m4,}
  \DataTypeTok{pred.labels =}\NormalTok{ pl,}
  \DataTypeTok{dv.labels =} \KeywordTok{c}\NormalTok{(}\StringTok{"Model1"}\NormalTok{, }\StringTok{"Model2"}\NormalTok{, }\StringTok{"Model3"}\NormalTok{, }\StringTok{"Model4"}\NormalTok{),}
  \DataTypeTok{show.ci =} \OtherTok{FALSE}\NormalTok{,}
  \DataTypeTok{show.p =} \OtherTok{FALSE}\NormalTok{,}
  \DataTypeTok{transform =} \OtherTok{NULL}
\NormalTok{)}
\end{Highlighting}
\end{Shaded}

~

Model1

Model2

Model3

Model4

Predictors

Estimates

Estimates

Log-Mean

Log-Odds

Intercept

87.15

9.83

-1.19

1.88

Age (Carer)

-0.21

0.01

0.01

Hours per Week

-0.28

0.02

0.00

Gender (Carer)

-0.39

0.43

0.01

0.63

Education: middle (Carer)

1.37

0.39

0.21

Education: high (Carer)

-1.64

0.64

0.31

Age (Older Person)

0.01

Barthel-Index

-0.03

Observations

821

879

840

815

R2 / adjusted R2

0.271 / 0.266

0.067 / 0.063

0.083 / 0.106

0.184 / 0.247

\begin{Shaded}
\begin{Highlighting}[]
\CommentTok{# Keep or remove coefficients from the table}
\CommentTok{# Using the terms- or rm.terms-argument allows us to explicitly show or }
\CommentTok{# remove specific coefficients from the table output.}

\KeywordTok{tab_model}\NormalTok{(m1, }\DataTypeTok{terms =} \KeywordTok{c}\NormalTok{(}\StringTok{"c160age"}\NormalTok{, }\StringTok{"c12hour"}\NormalTok{))}
\end{Highlighting}
\end{Shaded}

~

Total score BARTHEL INDEX

Predictors

Estimates

CI

p

carer'age

-0.21

-0.35~--~-0.07

0.004

average number of hoursof care per week

-0.28

-0.32~--~-0.24

\textless{}0.001

Observations

821

R2 / adjusted R2

0.271 / 0.266

\begin{Shaded}
\begin{Highlighting}[]
\CommentTok{# Note that the names of terms to keep or remove should match the coefficients}
\CommentTok{# names. }

\CommentTok{# For categorical predictors, one example would be, which will remove the}
\CommentTok{# terms c172code2 and c161sex2 from the summary, even though those two}
\CommentTok{# terms were still used to fit the final model:}

\KeywordTok{tab_model}\NormalTok{(m1, }\DataTypeTok{rm.terms =} \KeywordTok{c}\NormalTok{(}\StringTok{"c172code2"}\NormalTok{, }\StringTok{"c161sex2"}\NormalTok{))}
\end{Highlighting}
\end{Shaded}

~

Total score BARTHEL INDEX

Predictors

Estimates

CI

p

(Intercept)

87.15

77.98~--~96.32

\textless{}0.001

carer'age

-0.21

-0.35~--~-0.07

0.004

average number of hoursof care per week

-0.28

-0.32~--~-0.24

\textless{}0.001

high level of education

-1.64

-7.21~--~3.93

0.564

Observations

821

R2 / adjusted R2

0.271 / 0.266

\begin{Shaded}
\begin{Highlighting}[]
\CommentTok{# For How to format an Anova table output see:}

\CommentTok{# http://www.understandingdata.net/2017/05/11/anova-tables-in-r/}
\end{Highlighting}
\end{Shaded}


\end{document}
